\documentclass[9pt,letter]{article}
\usepackage{lmodern}
\usepackage{amssymb,amsmath}
\usepackage{ifxetex,ifluatex}
\usepackage{fixltx2e} % provides \textsubscript
\ifnum 0\ifxetex 1\fi\ifluatex 1\fi=0 % if pdftex
  \usepackage[T1]{fontenc}
  \usepackage[utf8]{inputenc}
\else % if luatex or xelatex
  \ifxetex
    \usepackage{mathspec}
  \else
    \usepackage{fontspec}
  \fi
  \defaultfontfeatures{Ligatures=TeX,Scale=MatchLowercase}
\fi
% use upquote if available, for straight quotes in verbatim environments
\IfFileExists{upquote.sty}{\usepackage{upquote}}{}
% use microtype if available
\IfFileExists{microtype.sty}{%
\usepackage{microtype}
\UseMicrotypeSet[protrusion]{basicmath} % disable protrusion for tt fonts
}{}
\usepackage[margin=.75in]{geometry}
\usepackage{hyperref}
\hypersetup{unicode=true,
            pdftitle={Assignment 2: Data Visualization},
            pdfborder={0 0 0},
            breaklinks=true}
\urlstyle{same}  % don't use monospace font for urls
\usepackage{graphicx,grffile}
\makeatletter
\def\maxwidth{\ifdim\Gin@nat@width>\linewidth\linewidth\else\Gin@nat@width\fi}
\def\maxheight{\ifdim\Gin@nat@height>\textheight\textheight\else\Gin@nat@height\fi}
\makeatother
% Scale images if necessary, so that they will not overflow the page
% margins by default, and it is still possible to overwrite the defaults
% using explicit options in \includegraphics[width, height, ...]{}
\setkeys{Gin}{width=\maxwidth,height=\maxheight,keepaspectratio}
\IfFileExists{parskip.sty}{%
\usepackage{parskip}
}{% else
\setlength{\parindent}{0pt}
\setlength{\parskip}{6pt plus 2pt minus 1pt}
}
\setlength{\emergencystretch}{3em}  % prevent overfull lines
\providecommand{\tightlist}{%
  \setlength{\itemsep}{0pt}\setlength{\parskip}{0pt}}
\setcounter{secnumdepth}{0}
% Redefines (sub)paragraphs to behave more like sections
\ifx\paragraph\undefined\else
\let\oldparagraph\paragraph
\renewcommand{\paragraph}[1]{\oldparagraph{#1}\mbox{}}
\fi
\ifx\subparagraph\undefined\else
\let\oldsubparagraph\subparagraph
\renewcommand{\subparagraph}[1]{\oldsubparagraph{#1}\mbox{}}
\fi

%%% Use protect on footnotes to avoid problems with footnotes in titles
\let\rmarkdownfootnote\footnote%
\def\footnote{\protect\rmarkdownfootnote}

%%% Change title format to be more compact
\usepackage{titling}

% Create subtitle command for use in maketitle
\newcommand{\subtitle}[1]{
  \posttitle{
    \begin{center}\large#1\end{center}
    }
}

\setlength{\droptitle}{-2em}
  \title{Assignment 2: Data Visualization}
  \pretitle{\vspace{\droptitle}\centering\huge}
  \posttitle{\par}
\subtitle{Due beginning of class: Monday September 25}
  \author{}
  \preauthor{}\postauthor{}
  \date{}
  \predate{}\postdate{}

\usepackage{graphicx}
\usepackage{color}
\usepackage{booktabs}
\usepackage{longtable}
\usepackage{array}
\usepackage{multirow}
\usepackage[table]{xcolor}
\usepackage{wrapfig}
\usepackage{float}
\usepackage{colortbl}
\usepackage{pdflscape}
\usepackage{tabu}
\usepackage{threeparttable}

\begin{document}
\maketitle

\begin{enumerate}
\def\labelenumi{\arabic{enumi}.}
\setcounter{enumi}{5}
\tightlist
\item
  \textbf{Graduate students} (bonus undergraduates): Suppose we have
  \(n\)-dimensional real and linearly independent vectors
  \({\bf x}_1, {\bf x}_2, \ldots , {\bf x}_p\) and \({\bf y}\). The
  vector \({\bf y}\) is the sum of two \(n\)-dimensional real vectors
  \({\bf \mu}\) and \({\bf r}\) \[ {\bf y} = {\bf \mu} + {\bf r} \]
  where \({\bf \mu}\) is restricted to be a linear combination of the
  vectors \({\bf x}_1, {\bf x}_2, \ldots , {\bf x}_p\). That is
  \[ {\bf \mu} = \theta_1 \times {\bf x}_1 + \theta_2 \times {\bf x}_2 + \cdots  + \theta_p \times{\bf x}_p\]
  for some unknown real constants
  \(\theta_1, \theta_2, \ldots, \theta_p\), or equivalently
  \[  {\bf \mu}  = {\bf X}{\bf \theta}\] where
  \({\bf X} = [ {\bf x}_1, \ldots, {\bf x}_p]\) is an \(n \times p\)
  matrix and \({\bf \theta} = (\theta_1, \theta_2, \ldots, \theta_p)^T\)
  is a \(p \times 1\) vector.
\end{enumerate}

\begin{enumerate}
\def\labelenumi{(\alph{enumi})}
\tightlist
\item
  (5 marks) For any \({\bf y}\), neither \({\bf \mu}\) nor \({\bf r}\)
  are uniquely defined. Suppose we choose particular vectors
  \(\widehat{\bf \mu}\), and \(\widehat{\bf r}\) (with
  \({\bf y} = \widehat{\bf \mu} + \widehat{\bf r}~\)) to be such that
  they are orthogonal to one another (whatever values any \(\theta_i\)
  take). That is, \(\widehat{\bf \mu}^T \widehat{\bf r} = 0\).
\end{enumerate}

Prove that this additional constraint implies that
\[\widehat{\bf \mu} = {\bf P} {\bf y} \] where
\({\bf P} = {\bf X} ({\bf X}^T {\bf X})^{-1}{\bf X}^T\) and hence show
that \(\widehat{\bf r} = ({\bf I}_n -{\bf P}){\bf y}\).

\(\widehat{\bf \mu}^T \widehat{\bf r} = 0\) (whatever values any
\(\theta_i\) take) \(\implies \widehat{\bf r}\) is orthogonal to
\({\bf x}_1, {\bf x}_2, \ldots, {\bf x}_p\). Hence: \[
\begin{array}{rcl}
{\bf X}^T \widehat{\bf r} &=& \begin{bmatrix}
                                {\bf x}_1^T\widehat{\bf r}\\[0.3em]
                                \cdots\\[0.3em]
                                {\bf x}_p^T\widehat{\bf r}
                              \end{bmatrix}\\
&=& {\bf 0}\\
{\bf X}({\bf X}^T{\bf X})^{-1}{\bf X}^T\widehat{\bf r} &=& {\bf X}({\bf X}^T{\bf X})^{-1}{\bf 0}\\
{\bf P}\widehat{\bf r} &=& {\bf 0}\\
{\bf P}{\bf y} - {\bf P}\widehat{\bf r} &=& {\bf P}{\bf y}\\
{\bf P}\widehat{\bf \mu} &=& {\bf P}{\bf y}
\end{array}
\] But we can simplify \({\bf P}{\bf u}\) to: \[
\begin{array}{rcl}
{\bf P}{\bf \mu} &=& {\bf X}({\bf X}^T{\bf X})^{-1}{\bf X}^T{\bf \mu}\\
&=& {\bf X}({\bf X}^T{\bf X})^{-1}{\bf X}^T{\bf X}{\bf \theta}\\
&=& {\bf X}{\bf I}_p{\bf \theta}\\
&=& {\bf X}{\bf \theta}\\
&=& \widehat{\bf \mu}
\end{array}
\] \(\therefore \widehat{\bf \mu} = {\bf P}{\bf y}\). For the second
proof:

\[
\begin{array}{rcl}
{\bf P}{\bf y} &=& \widehat{\bf \mu}\\
&=& {\bf y} - \widehat{\bf r}\\
\widehat{\bf r} &=& {\bf y} - {\bf P}{\bf y}\\
&=& ({\bf I}_n -{\bf P}){\bf y}
\end{array}
\]

\begin{enumerate}
\def\labelenumi{(\alph{enumi})}
\setcounter{enumi}{1}
\tightlist
\item
  (2 marks) Show that \({\bf P}\) is an idempotent matrix, that is that
  \({\bf P}^2={\bf P}\).
\end{enumerate}

\[
\begin{array}{rcl}
{\bf P}^2 &=& {\bf X} ({\bf X}^T {\bf X})^{-1}{\bf X}^T{\bf X} ({\bf X}^T {\bf X})^{-1}{\bf X}^T\\
&=& {\bf X}({\bf X}^T {\bf X})^{-1}{\bf I}_p{\bf X}^T\\
&=& {\bf X}({\bf X}^T {\bf X})^{-1}{\bf X}^T\\
&=& {\bf P}
\end{array}
\]

\(\therefore {\bf P}\) is an idempotent matrix.

\begin{enumerate}
\def\labelenumi{(\alph{enumi})}
\setcounter{enumi}{2}
\tightlist
\item
  (2 marks) Show that if \({\bf P}\) is an idempotent matrix, then so
  must be \$ (\{\bf I\}\_n -\{\bf P\})\$.
\end{enumerate}

Assume that \({\bf P}\) is an idempotent matrix.

\[
\begin{array}{rcl}
({\bf I}_n -{\bf P})^2 &= ({\bf I}_n -{\bf P})({\bf I}_n -{\bf P}) \\
&= {\bf I}_n^2 -{\bf I}_n{\bf P}-{\bf P}{\bf I}_n+{\bf P}^2 \\
&= {\bf I}_n-{\bf P}-{\bf P}+{\bf P} & \text{since } {\bf P} \text{ is idempotent} \\
&= {\bf I}_n-{\bf P}
\end{array}
\]

\(\therefore ({\bf I}_n -{\bf P})\) is an idempotent matrix.

\begin{enumerate}
\def\labelenumi{(\alph{enumi})}
\setcounter{enumi}{3}
\tightlist
\item
  (2 marks) Show that \(\widehat{\bf r}\) is in fact orthogonal to
  \(\widehat{\bf \mu}\).
\end{enumerate}

We first show that P is symmetric:

\[
\begin{array}{rcl}
{\bf P}^T &=& ({\bf X} ({\bf X}^T {\bf X})^{-1}{\bf X}^T)^T\\
&=& ({\bf X}^T)^T({\bf X} ({\bf X}^T {\bf X})^{-1})^T\\
&=& {\bf X}(({\bf X}^T {\bf X})^{-1})^T{\bf X}^T\\
&=& {\bf X}(({\bf X}^T {\bf X})^T)^{-1}{\bf X}^T\\
&=& {\bf X}({\bf X}^T {\bf X})^{-1}{\bf X}^T\\
&=& {\bf P}
\end{array}
\]

\(\therefore {\bf P}\) is symmetric.

\[
\begin{array}{rcl}
\widehat{\bf \mu}^T \widehat{\bf r} &= & \widehat{\bf \mu} \cdot \widehat{\bf r}\\
&= & {\bf P}{\bf y} \cdot ({\bf I}_n-{\bf P}){\bf y}\\
&= & {\bf y} \cdot{\bf P} ({\bf I}_n-{\bf P}){\bf y} \\
&= & {\bf y}^T({\bf P}-{\bf P}^2){\bf y}\\
&= & {\bf y}^T{\bf 0}{\bf y} \\
&= & {\bf 0} \\
\end{array}
\]

\(\therefore \widehat{\bf \mu}\) is orthogonal to \(\widehat{\bf r}\).

\begin{enumerate}
\def\labelenumi{(\alph{enumi})}
\setcounter{enumi}{4}
\tightlist
\item
  (5 marks) Show that \(\widehat{\bf \mu} = {\bf P} {\bf y}\) is the
  choice of \({\bf \mu}\) which minimizes the squared length of
  \({\bf r}\). That is it minimizes
  \({\bf r}^T{\bf r}=({\bf y} - {\bf \mu})^T({\bf y} - {\bf \mu})\).
\end{enumerate}


\end{document}
