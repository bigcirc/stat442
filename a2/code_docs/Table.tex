\documentclass[]{article}
\usepackage{lmodern}
\usepackage{amssymb,amsmath}
\usepackage{ifxetex,ifluatex}
\usepackage{fixltx2e} % provides \textsubscript
\ifnum 0\ifxetex 1\fi\ifluatex 1\fi=0 % if pdftex
  \usepackage[T1]{fontenc}
  \usepackage[utf8]{inputenc}
\else % if luatex or xelatex
  \ifxetex
    \usepackage{mathspec}
  \else
    \usepackage{fontspec}
  \fi
  \defaultfontfeatures{Ligatures=TeX,Scale=MatchLowercase}
\fi
% use upquote if available, for straight quotes in verbatim environments
\IfFileExists{upquote.sty}{\usepackage{upquote}}{}
% use microtype if available
\IfFileExists{microtype.sty}{%
\usepackage{microtype}
\UseMicrotypeSet[protrusion]{basicmath} % disable protrusion for tt fonts
}{}
\usepackage[margin=1in]{geometry}
\usepackage{hyperref}
\hypersetup{unicode=true,
            pdftitle={Tables in knitr},
            pdfauthor={R.W. Oldford},
            pdfborder={0 0 0},
            breaklinks=true}
\urlstyle{same}  % don't use monospace font for urls
\usepackage{color}
\usepackage{fancyvrb}
\newcommand{\VerbBar}{|}
\newcommand{\VERB}{\Verb[commandchars=\\\{\}]}
\DefineVerbatimEnvironment{Highlighting}{Verbatim}{commandchars=\\\{\}}
% Add ',fontsize=\small' for more characters per line
\usepackage{framed}
\definecolor{shadecolor}{RGB}{248,248,248}
\newenvironment{Shaded}{\begin{snugshade}}{\end{snugshade}}
\newcommand{\KeywordTok}[1]{\textcolor[rgb]{0.13,0.29,0.53}{\textbf{#1}}}
\newcommand{\DataTypeTok}[1]{\textcolor[rgb]{0.13,0.29,0.53}{#1}}
\newcommand{\DecValTok}[1]{\textcolor[rgb]{0.00,0.00,0.81}{#1}}
\newcommand{\BaseNTok}[1]{\textcolor[rgb]{0.00,0.00,0.81}{#1}}
\newcommand{\FloatTok}[1]{\textcolor[rgb]{0.00,0.00,0.81}{#1}}
\newcommand{\ConstantTok}[1]{\textcolor[rgb]{0.00,0.00,0.00}{#1}}
\newcommand{\CharTok}[1]{\textcolor[rgb]{0.31,0.60,0.02}{#1}}
\newcommand{\SpecialCharTok}[1]{\textcolor[rgb]{0.00,0.00,0.00}{#1}}
\newcommand{\StringTok}[1]{\textcolor[rgb]{0.31,0.60,0.02}{#1}}
\newcommand{\VerbatimStringTok}[1]{\textcolor[rgb]{0.31,0.60,0.02}{#1}}
\newcommand{\SpecialStringTok}[1]{\textcolor[rgb]{0.31,0.60,0.02}{#1}}
\newcommand{\ImportTok}[1]{#1}
\newcommand{\CommentTok}[1]{\textcolor[rgb]{0.56,0.35,0.01}{\textit{#1}}}
\newcommand{\DocumentationTok}[1]{\textcolor[rgb]{0.56,0.35,0.01}{\textbf{\textit{#1}}}}
\newcommand{\AnnotationTok}[1]{\textcolor[rgb]{0.56,0.35,0.01}{\textbf{\textit{#1}}}}
\newcommand{\CommentVarTok}[1]{\textcolor[rgb]{0.56,0.35,0.01}{\textbf{\textit{#1}}}}
\newcommand{\OtherTok}[1]{\textcolor[rgb]{0.56,0.35,0.01}{#1}}
\newcommand{\FunctionTok}[1]{\textcolor[rgb]{0.00,0.00,0.00}{#1}}
\newcommand{\VariableTok}[1]{\textcolor[rgb]{0.00,0.00,0.00}{#1}}
\newcommand{\ControlFlowTok}[1]{\textcolor[rgb]{0.13,0.29,0.53}{\textbf{#1}}}
\newcommand{\OperatorTok}[1]{\textcolor[rgb]{0.81,0.36,0.00}{\textbf{#1}}}
\newcommand{\BuiltInTok}[1]{#1}
\newcommand{\ExtensionTok}[1]{#1}
\newcommand{\PreprocessorTok}[1]{\textcolor[rgb]{0.56,0.35,0.01}{\textit{#1}}}
\newcommand{\AttributeTok}[1]{\textcolor[rgb]{0.77,0.63,0.00}{#1}}
\newcommand{\RegionMarkerTok}[1]{#1}
\newcommand{\InformationTok}[1]{\textcolor[rgb]{0.56,0.35,0.01}{\textbf{\textit{#1}}}}
\newcommand{\WarningTok}[1]{\textcolor[rgb]{0.56,0.35,0.01}{\textbf{\textit{#1}}}}
\newcommand{\AlertTok}[1]{\textcolor[rgb]{0.94,0.16,0.16}{#1}}
\newcommand{\ErrorTok}[1]{\textcolor[rgb]{0.64,0.00,0.00}{\textbf{#1}}}
\newcommand{\NormalTok}[1]{#1}
\usepackage{longtable,booktabs}
\usepackage{graphicx,grffile}
\makeatletter
\def\maxwidth{\ifdim\Gin@nat@width>\linewidth\linewidth\else\Gin@nat@width\fi}
\def\maxheight{\ifdim\Gin@nat@height>\textheight\textheight\else\Gin@nat@height\fi}
\makeatother
% Scale images if necessary, so that they will not overflow the page
% margins by default, and it is still possible to overwrite the defaults
% using explicit options in \includegraphics[width, height, ...]{}
\setkeys{Gin}{width=\maxwidth,height=\maxheight,keepaspectratio}
\IfFileExists{parskip.sty}{%
\usepackage{parskip}
}{% else
\setlength{\parindent}{0pt}
\setlength{\parskip}{6pt plus 2pt minus 1pt}
}
\setlength{\emergencystretch}{3em}  % prevent overfull lines
\providecommand{\tightlist}{%
  \setlength{\itemsep}{0pt}\setlength{\parskip}{0pt}}
\setcounter{secnumdepth}{0}
% Redefines (sub)paragraphs to behave more like sections
\ifx\paragraph\undefined\else
\let\oldparagraph\paragraph
\renewcommand{\paragraph}[1]{\oldparagraph{#1}\mbox{}}
\fi
\ifx\subparagraph\undefined\else
\let\oldsubparagraph\subparagraph
\renewcommand{\subparagraph}[1]{\oldsubparagraph{#1}\mbox{}}
\fi

%%% Use protect on footnotes to avoid problems with footnotes in titles
\let\rmarkdownfootnote\footnote%
\def\footnote{\protect\rmarkdownfootnote}

%%% Change title format to be more compact
\usepackage{titling}

% Create subtitle command for use in maketitle
\newcommand{\subtitle}[1]{
  \posttitle{
    \begin{center}\large#1\end{center}
    }
}

\setlength{\droptitle}{-2em}
  \title{Tables in knitr}
  \pretitle{\vspace{\droptitle}\centering\huge}
  \posttitle{\par}
  \author{R.W. Oldford}
  \preauthor{\centering\large\emph}
  \postauthor{\par}
  \date{}
  \predate{}\postdate{}

\usepackage{booktabs}
\usepackage{longtable}
\usepackage{array}
\usepackage{multirow}
\usepackage[table]{xcolor}
\usepackage{wrapfig}
\usepackage{float}
\usepackage{colortbl}
\usepackage{pdflscape}
\usepackage{tabu}
\usepackage{threeparttable}
\usepackage{graphicx}
\usepackage{color}
\usepackage{booktabs}
\usepackage{longtable}
\usepackage{array}
\usepackage{multirow}
\usepackage[table]{xcolor}
\usepackage{wrapfig}
\usepackage{float}
\usepackage{colortbl}
\usepackage{pdflscape}
\usepackage{tabu}
\usepackage{threeparttable}

\begin{document}
\maketitle

In the \texttt{knitr} package there is a function called \texttt{kable}
which is available to help produce some nicely formatted tables from
RMarkdown (for example). See \texttt{help(kable)} for details (you may
need to execute \texttt{library(knitr)} first).

Here we will illustrate some of its functionality by using part of a
dataset from \texttt{R}.

\begin{Shaded}
\begin{Highlighting}[]
\KeywordTok{library}\NormalTok{(knitr)}
\NormalTok{data <-}\StringTok{ }\KeywordTok{head}\NormalTok{(mtcars)}
\KeywordTok{kable}\NormalTok{(data, }\DataTypeTok{caption=}\StringTok{"Motor trends' car data"}\NormalTok{)}
\end{Highlighting}
\end{Shaded}

\begin{longtable}[]{@{}lrrrrrrrrrrr@{}}
\caption{Motor trends' car data}\tabularnewline
\toprule
& mpg & cyl & disp & hp & drat & wt & qsec & vs & am & gear &
carb\tabularnewline
\midrule
\endfirsthead
\toprule
& mpg & cyl & disp & hp & drat & wt & qsec & vs & am & gear &
carb\tabularnewline
\midrule
\endhead
Mazda RX4 & 21.0 & 6 & 160 & 110 & 3.90 & 2.620 & 16.46 & 0 & 1 & 4 &
4\tabularnewline
Mazda RX4 Wag & 21.0 & 6 & 160 & 110 & 3.90 & 2.875 & 17.02 & 0 & 1 & 4
& 4\tabularnewline
Datsun 710 & 22.8 & 4 & 108 & 93 & 3.85 & 2.320 & 18.61 & 1 & 1 & 4 &
1\tabularnewline
Hornet 4 Drive & 21.4 & 6 & 258 & 110 & 3.08 & 3.215 & 19.44 & 1 & 0 & 3
& 1\tabularnewline
Hornet Sportabout & 18.7 & 8 & 360 & 175 & 3.15 & 3.440 & 17.02 & 0 & 0
& 3 & 2\tabularnewline
Valiant & 18.1 & 6 & 225 & 105 & 2.76 & 3.460 & 20.22 & 1 & 0 & 3 &
1\tabularnewline
\bottomrule
\end{longtable}

\begin{Shaded}
\begin{Highlighting}[]
\KeywordTok{library}\NormalTok{(kableExtra)}
\NormalTok{dt <-}\StringTok{ }\NormalTok{mtcars[}\DecValTok{1}\OperatorTok{:}\DecValTok{5}\NormalTok{, }\DecValTok{1}\OperatorTok{:}\DecValTok{6}\NormalTok{]}
\NormalTok{dt}
\end{Highlighting}
\end{Shaded}

\begin{verbatim}
##                    mpg cyl disp  hp drat    wt
## Mazda RX4         21.0   6  160 110 3.90 2.620
## Mazda RX4 Wag     21.0   6  160 110 3.90 2.875
## Datsun 710        22.8   4  108  93 3.85 2.320
## Hornet 4 Drive    21.4   6  258 110 3.08 3.215
## Hornet Sportabout 18.7   8  360 175 3.15 3.440
\end{verbatim}

\begin{Shaded}
\begin{Highlighting}[]
\KeywordTok{kable}\NormalTok{(dt, }\StringTok{"latex"}\NormalTok{) }\OperatorTok
\StringTok{  }\KeywordTok{kable_styling}\NormalTok{(}\StringTok{"striped"}\NormalTok{) }\OperatorTok
\StringTok{  }\KeywordTok{add_header_above}\NormalTok{(}\KeywordTok{c}\NormalTok{(}\StringTok{" "}\NormalTok{ =}\StringTok{ }\DecValTok{1}\NormalTok{, }\StringTok{"Group 1"}\NormalTok{ =}\StringTok{ }\DecValTok{2}\NormalTok{, }\StringTok{"Group 2"}\NormalTok{ =}\StringTok{ }\DecValTok{2}\NormalTok{, }\StringTok{"Group 3"}\NormalTok{ =}\StringTok{ }\DecValTok{2}\NormalTok{))}
\end{Highlighting}
\end{Shaded}

\begin{table}[!h]
\centering
\begin{tabular}{l|r|r|r|r|r|r}
\hline
\multicolumn{1}{c|}{ } & \multicolumn{2}{|c|}{Group 1} & \multicolumn{2}{|c|}{Group 2} & \multicolumn{2}{|c}{Group 3} \\
\cline{2-3} \cline{4-5} \cline{6-7}
  & mpg & cyl & disp & hp & drat & wt\\
\hline
Mazda RX4 & 21.0 & 6 & 160 & 110 & 3.90 & 2.620\\
\hline
Mazda RX4 Wag & 21.0 & 6 & 160 & 110 & 3.90 & 2.875\\
\hline
Datsun 710 & 22.8 & 4 & 108 & 93 & 3.85 & 2.320\\
\hline
Hornet 4 Drive & 21.4 & 6 & 258 & 110 & 3.08 & 3.215\\
\hline
Hornet Sportabout & 18.7 & 8 & 360 & 175 & 3.15 & 3.440\\
\hline
\end{tabular}
\end{table}

\texttt{r\ \ \ \ \ cleanWater\ =\ c(82,\ 66)\ \ \ \ \ waterData\ =\ data.frame(c(\textquotesingle{}Satisfactory\textquotesingle{},\ \textquotesingle{}Unsatisfactory\textquotesingle{}),\ rbind(cleanWater,\ 100-cleanWater))\ \ \ \ \ colnames(waterData)\ =\ c(\textquotesingle{}\textquotesingle{},\ "(\%)",\ "(\%)")\ \ \ \ \ rownames(waterData)\ =\ NULL\ \ \ \ \ kable(waterData,\ "latex",\ booktabs\ =\ T)\ \%\textgreater{}\%\ \ \ \ \ \ \ kable\_styling()\ \%\textgreater{}\%\ \ \ \ \ \ \ add\_header\_above(c("Water\ Source"\ =\ 1,\ "2000"\ =\ 1,\ "2015"\ =\ 1))\ \%\textgreater{}\%\ \ \ \ \ \ \ add\_header\_above(c("\ "\ =\ 1,\ "Year"\ =\ 2))}

\begin{verbatim}
\begin{table}[!h]
\centering
\begin{tabular}{lrr}
\toprule
\multicolumn{1}{c}{ } & \multicolumn{2}{c}{Year} \\
\cmidrule(l{2pt}r{2pt}){2-3}
\multicolumn{1}{c}{Water Source} & \multicolumn{1}{c}{2000} & \multicolumn{1}{c}{2015} \\
\cmidrule(l{2pt}r{2pt}){1-1} \cmidrule(l{2pt}r{2pt}){2-2} \cmidrule(l{2pt}r{2pt}){3-3}
 & (\%) & (\%)\\
\midrule
Satisfactory & 82 & 66\\
Unsatisfactory & 18 & 34\\
\bottomrule
\end{tabular}
\end{table}
\end{verbatim}

Note that \texttt{kable} already displays the numbers so that they line
up positionally. Some of the functionality of \texttt{kable} can be seen
in the next few examples:

\begin{Shaded}
\begin{Highlighting}[]
\CommentTok{# swap rows and columns (transpose of data = t(data) )}
\CommentTok{# Note that it matches the number of decimal places within each column }
\CommentTok{# so that the numbers line up vertically.}
\KeywordTok{kable}\NormalTok{(}\KeywordTok{t}\NormalTok{(data))  }
\end{Highlighting}
\end{Shaded}

\begin{longtable}[]{@{}lrrrrrr@{}}
\toprule
& Mazda RX4 & Mazda RX4 Wag & Datsun 710 & Hornet 4 Drive & Hornet
Sportabout & Valiant\tabularnewline
\midrule
\endhead
mpg & 21.00 & 21.000 & 22.80 & 21.400 & 18.70 & 18.10\tabularnewline
cyl & 6.00 & 6.000 & 4.00 & 6.000 & 8.00 & 6.00\tabularnewline
disp & 160.00 & 160.000 & 108.00 & 258.000 & 360.00 &
225.00\tabularnewline
hp & 110.00 & 110.000 & 93.00 & 110.000 & 175.00 & 105.00\tabularnewline
drat & 3.90 & 3.900 & 3.85 & 3.080 & 3.15 & 2.76\tabularnewline
wt & 2.62 & 2.875 & 2.32 & 3.215 & 3.44 & 3.46\tabularnewline
qsec & 16.46 & 17.020 & 18.61 & 19.440 & 17.02 & 20.22\tabularnewline
vs & 0.00 & 0.000 & 1.00 & 1.000 & 0.00 & 1.00\tabularnewline
am & 1.00 & 1.000 & 1.00 & 0.000 & 0.00 & 0.00\tabularnewline
gear & 4.00 & 4.000 & 4.00 & 3.000 & 3.00 & 3.00\tabularnewline
carb & 4.00 & 4.000 & 1.00 & 1.000 & 2.00 & 1.00\tabularnewline
\bottomrule
\end{longtable}

\begin{Shaded}
\begin{Highlighting}[]
\CommentTok{# swap rows and columns and round so that there are no (digits = 0) decimal places}
\KeywordTok{kable}\NormalTok{(}\KeywordTok{t}\NormalTok{(data), }\DataTypeTok{digits=}\DecValTok{0}\NormalTok{)}
\end{Highlighting}
\end{Shaded}

\begin{longtable}[]{@{}lrrrrrr@{}}
\toprule
& Mazda RX4 & Mazda RX4 Wag & Datsun 710 & Hornet 4 Drive & Hornet
Sportabout & Valiant\tabularnewline
\midrule
\endhead
mpg & 21 & 21 & 23 & 21 & 19 & 18\tabularnewline
cyl & 6 & 6 & 4 & 6 & 8 & 6\tabularnewline
disp & 160 & 160 & 108 & 258 & 360 & 225\tabularnewline
hp & 110 & 110 & 93 & 110 & 175 & 105\tabularnewline
drat & 4 & 4 & 4 & 3 & 3 & 3\tabularnewline
wt & 3 & 3 & 2 & 3 & 3 & 3\tabularnewline
qsec & 16 & 17 & 19 & 19 & 17 & 20\tabularnewline
vs & 0 & 0 & 1 & 1 & 0 & 1\tabularnewline
am & 1 & 1 & 1 & 0 & 0 & 0\tabularnewline
gear & 4 & 4 & 4 & 3 & 3 & 3\tabularnewline
carb & 4 & 4 & 1 & 1 & 2 & 1\tabularnewline
\bottomrule
\end{longtable}

\begin{Shaded}
\begin{Highlighting}[]
\CommentTok{# Changing alignment in columns here}
\KeywordTok{kable}\NormalTok{(}\KeywordTok{t}\NormalTok{(data), }\DataTypeTok{digits =} \DecValTok{0}\NormalTok{, }\DataTypeTok{align=}\StringTok{"lcrlcr"}\NormalTok{)}
\end{Highlighting}
\end{Shaded}

\begin{longtable}[]{@{}llcrlcr@{}}
\toprule
& Mazda RX4 & Mazda RX4 Wag & Datsun 710 & Hornet 4 Drive & Hornet
Sportabout & Valiant\tabularnewline
\midrule
\endhead
mpg & 21 & 21 & 23 & 21 & 19 & 18\tabularnewline
cyl & 6 & 6 & 4 & 6 & 8 & 6\tabularnewline
disp & 160 & 160 & 108 & 258 & 360 & 225\tabularnewline
hp & 110 & 110 & 93 & 110 & 175 & 105\tabularnewline
drat & 4 & 4 & 4 & 3 & 3 & 3\tabularnewline
wt & 3 & 3 & 2 & 3 & 3 & 3\tabularnewline
qsec & 16 & 17 & 19 & 19 & 17 & 20\tabularnewline
vs & 0 & 0 & 1 & 1 & 0 & 1\tabularnewline
am & 1 & 1 & 1 & 0 & 0 & 0\tabularnewline
gear & 4 & 4 & 4 & 3 & 3 & 3\tabularnewline
carb & 4 & 4 & 1 & 1 & 2 & 1\tabularnewline
\bottomrule
\end{longtable}

\begin{Shaded}
\begin{Highlighting}[]
\CommentTok{# back to the original rows and columns but reducing the number of digits displayed}
\KeywordTok{kable}\NormalTok{(data}\OperatorTok{/}\DecValTok{10}\NormalTok{, }\DataTypeTok{digits=}\DecValTok{0}\NormalTok{)}
\end{Highlighting}
\end{Shaded}

\begin{longtable}[]{@{}lrrrrrrrrrrr@{}}
\toprule
& mpg & cyl & disp & hp & drat & wt & qsec & vs & am & gear &
carb\tabularnewline
\midrule
\endhead
Mazda RX4 & 2 & 1 & 16 & 11 & 0 & 0 & 2 & 0 & 0 & 0 & 0\tabularnewline
Mazda RX4 Wag & 2 & 1 & 16 & 11 & 0 & 0 & 2 & 0 & 0 & 0 &
0\tabularnewline
Datsun 710 & 2 & 0 & 11 & 9 & 0 & 0 & 2 & 0 & 0 & 0 & 0\tabularnewline
Hornet 4 Drive & 2 & 1 & 26 & 11 & 0 & 0 & 2 & 0 & 0 & 0 &
0\tabularnewline
Hornet Sportabout & 2 & 1 & 36 & 18 & 0 & 0 & 2 & 0 & 0 & 0 &
0\tabularnewline
Valiant & 2 & 1 & 22 & 10 & 0 & 0 & 2 & 0 & 0 & 0 & 0\tabularnewline
\bottomrule
\end{longtable}

\begin{Shaded}
\begin{Highlighting}[]
\CommentTok{# No rownames}
\KeywordTok{kable}\NormalTok{(data, }\DataTypeTok{row.names =} \OtherTok{FALSE}\NormalTok{)}
\end{Highlighting}
\end{Shaded}

\begin{longtable}[]{@{}rrrrrrrrrrr@{}}
\toprule
mpg & cyl & disp & hp & drat & wt & qsec & vs & am & gear &
carb\tabularnewline
\midrule
\endhead
21.0 & 6 & 160 & 110 & 3.90 & 2.620 & 16.46 & 0 & 1 & 4 &
4\tabularnewline
21.0 & 6 & 160 & 110 & 3.90 & 2.875 & 17.02 & 0 & 1 & 4 &
4\tabularnewline
22.8 & 4 & 108 & 93 & 3.85 & 2.320 & 18.61 & 1 & 1 & 4 &
1\tabularnewline
21.4 & 6 & 258 & 110 & 3.08 & 3.215 & 19.44 & 1 & 0 & 3 &
1\tabularnewline
18.7 & 8 & 360 & 175 & 3.15 & 3.440 & 17.02 & 0 & 0 & 3 &
2\tabularnewline
18.1 & 6 & 225 & 105 & 2.76 & 3.460 & 20.22 & 1 & 0 & 3 &
1\tabularnewline
\bottomrule
\end{longtable}

\begin{Shaded}
\begin{Highlighting}[]
\CommentTok{# To actually change the row names we have to work on the dataset}
\NormalTok{orig.rownames <-}\StringTok{ }\KeywordTok{rownames}\NormalTok{(data)  }\CommentTok{# save the rownames}
\KeywordTok{rownames}\NormalTok{(data) <-}\StringTok{ }\NormalTok{LETTERS[}\DecValTok{1}\OperatorTok{:}\KeywordTok{nrow}\NormalTok{(data)]}
\KeywordTok{kable}\NormalTok{(data, }\DataTypeTok{digits =} \DecValTok{0}\NormalTok{)}
\end{Highlighting}
\end{Shaded}

\begin{longtable}[]{@{}lrrrrrrrrrrr@{}}
\toprule
& mpg & cyl & disp & hp & drat & wt & qsec & vs & am & gear &
carb\tabularnewline
\midrule
\endhead
A & 21 & 6 & 160 & 110 & 4 & 3 & 16 & 0 & 1 & 4 & 4\tabularnewline
B & 21 & 6 & 160 & 110 & 4 & 3 & 17 & 0 & 1 & 4 & 4\tabularnewline
C & 23 & 4 & 108 & 93 & 4 & 2 & 19 & 1 & 1 & 4 & 1\tabularnewline
D & 21 & 6 & 258 & 110 & 3 & 3 & 19 & 1 & 0 & 3 & 1\tabularnewline
E & 19 & 8 & 360 & 175 & 3 & 3 & 17 & 0 & 0 & 3 & 2\tabularnewline
F & 18 & 6 & 225 & 105 & 3 & 3 & 20 & 1 & 0 & 3 & 1\tabularnewline
\bottomrule
\end{longtable}

\begin{Shaded}
\begin{Highlighting}[]
\CommentTok{# And set the rownames back}
\KeywordTok{rownames}\NormalTok{(data) <-}\StringTok{ }\NormalTok{orig.rownames}

\CommentTok{# Changing the colnames could have been done in the same way using colnames(data)}
\CommentTok{# or through the kable function}
\KeywordTok{kable}\NormalTok{(data, }\DataTypeTok{digits =} \DecValTok{0}\NormalTok{, }\DataTypeTok{col.names =}\NormalTok{ LETTERS[}\DecValTok{1}\OperatorTok{:}\KeywordTok{ncol}\NormalTok{(data)])}
\end{Highlighting}
\end{Shaded}

\begin{longtable}[]{@{}lrrrrrrrrrrr@{}}
\toprule
& A & B & C & D & E & F & G & H & I & J & K\tabularnewline
\midrule
\endhead
Mazda RX4 & 21 & 6 & 160 & 110 & 4 & 3 & 16 & 0 & 1 & 4 &
4\tabularnewline
Mazda RX4 Wag & 21 & 6 & 160 & 110 & 4 & 3 & 17 & 0 & 1 & 4 &
4\tabularnewline
Datsun 710 & 23 & 4 & 108 & 93 & 4 & 2 & 19 & 1 & 1 & 4 &
1\tabularnewline
Hornet 4 Drive & 21 & 6 & 258 & 110 & 3 & 3 & 19 & 1 & 0 & 3 &
1\tabularnewline
Hornet Sportabout & 19 & 8 & 360 & 175 & 3 & 3 & 17 & 0 & 0 & 3 &
2\tabularnewline
Valiant & 18 & 6 & 225 & 105 & 3 & 3 & 20 & 1 & 0 & 3 & 1\tabularnewline
\bottomrule
\end{longtable}

It's also important to be able to manipulate the data before creating
the table.

Here we will add column means to the data and then sort the cols in
ascending order of column means.

\begin{Shaded}
\begin{Highlighting}[]
\CommentTok{# bind a row (rbind) to data of the colMeans }
\NormalTok{data1 <-}\StringTok{ }\KeywordTok{rbind}\NormalTok{(data, }\KeywordTok{colMeans}\NormalTok{(data))}
\KeywordTok{rownames}\NormalTok{(data1) <-}\StringTok{ }\KeywordTok{c}\NormalTok{(}\KeywordTok{rownames}\NormalTok{(data), }\StringTok{"Average"}\NormalTok{)}
\NormalTok{colOrder <-}\StringTok{ }\KeywordTok{order}\NormalTok{(}\KeywordTok{colMeans}\NormalTok{(data))}

\CommentTok{# The column means}
\KeywordTok{colMeans}\NormalTok{(data)}
\end{Highlighting}
\end{Shaded}

\begin{verbatim}
##        mpg        cyl       disp         hp       drat         wt 
##  20.500000   6.000000 211.833333 117.166667   3.440000   2.988333 
##       qsec         vs         am       gear       carb 
##  18.128333   0.500000   0.500000   3.500000   2.166667
\end{verbatim}

\begin{Shaded}
\begin{Highlighting}[]
\CommentTok{# the order from smallest to largest}
\KeywordTok{order}\NormalTok{(}\KeywordTok{colMeans}\NormalTok{(data))}
\end{Highlighting}
\end{Shaded}

\begin{verbatim}
##  [1]  8  9 11  6  5 10  2  7  1  4  3
\end{verbatim}

\begin{Shaded}
\begin{Highlighting}[]
\CommentTok{# So we reorder the data as}

\NormalTok{colorder <-}\StringTok{ }\KeywordTok{order}\NormalTok{(}\KeywordTok{colMeans}\NormalTok{(data))}
\CommentTok{# With the column averages}
\KeywordTok{kable}\NormalTok{(data1[,colorder], }\DataTypeTok{digits =} \DecValTok{0}\NormalTok{)}
\end{Highlighting}
\end{Shaded}

\begin{longtable}[]{@{}lrrrrrrrrrrr@{}}
\toprule
& vs & am & carb & wt & drat & gear & cyl & qsec & mpg & hp &
disp\tabularnewline
\midrule
\endhead
Mazda RX4 & 0 & 1 & 4 & 3 & 4 & 4 & 6 & 16 & 21 & 110 &
160\tabularnewline
Mazda RX4 Wag & 0 & 1 & 4 & 3 & 4 & 4 & 6 & 17 & 21 & 110 &
160\tabularnewline
Datsun 710 & 1 & 1 & 1 & 2 & 4 & 4 & 4 & 19 & 23 & 93 &
108\tabularnewline
Hornet 4 Drive & 1 & 0 & 1 & 3 & 3 & 3 & 6 & 19 & 21 & 110 &
258\tabularnewline
Hornet Sportabout & 0 & 0 & 2 & 3 & 3 & 3 & 8 & 17 & 19 & 175 &
360\tabularnewline
Valiant & 1 & 0 & 1 & 3 & 3 & 3 & 6 & 20 & 18 & 105 & 225\tabularnewline
Average & 0 & 0 & 2 & 3 & 3 & 4 & 6 & 18 & 20 & 117 & 212\tabularnewline
\bottomrule
\end{longtable}

\begin{Shaded}
\begin{Highlighting}[]
\CommentTok{# Without the column averages}
\KeywordTok{kable}\NormalTok{(data[,colorder], }\DataTypeTok{digits =} \DecValTok{0}\NormalTok{)}
\end{Highlighting}
\end{Shaded}

\begin{longtable}[]{@{}lrrrrrrrrrrr@{}}
\toprule
& vs & am & carb & wt & drat & gear & cyl & qsec & mpg & hp &
disp\tabularnewline
\midrule
\endhead
Mazda RX4 & 0 & 1 & 4 & 3 & 4 & 4 & 6 & 16 & 21 & 110 &
160\tabularnewline
Mazda RX4 Wag & 0 & 1 & 4 & 3 & 4 & 4 & 6 & 17 & 21 & 110 &
160\tabularnewline
Datsun 710 & 1 & 1 & 1 & 2 & 4 & 4 & 4 & 19 & 23 & 93 &
108\tabularnewline
Hornet 4 Drive & 1 & 0 & 1 & 3 & 3 & 3 & 6 & 19 & 21 & 110 &
258\tabularnewline
Hornet Sportabout & 0 & 0 & 2 & 3 & 3 & 3 & 8 & 17 & 19 & 175 &
360\tabularnewline
Valiant & 1 & 0 & 1 & 3 & 3 & 3 & 6 & 20 & 18 & 105 & 225\tabularnewline
\bottomrule
\end{longtable}

\begin{Shaded}
\begin{Highlighting}[]
\CommentTok{# And now order the rows in descending order of means.}
\NormalTok{roworder <-}\StringTok{ }\KeywordTok{order}\NormalTok{(}\KeywordTok{rowMeans}\NormalTok{(data), }\DataTypeTok{decreasing =} \OtherTok{TRUE}\NormalTok{)}
\CommentTok{# First order the table as desired}
\NormalTok{newtable <-}\StringTok{ }\NormalTok{data[, colorder][roworder,]}
\KeywordTok{kable}\NormalTok{(newtable, }\DataTypeTok{digits =} \DecValTok{0}\NormalTok{)}
\end{Highlighting}
\end{Shaded}

\begin{longtable}[]{@{}lrrrrrrrrrrr@{}}
\toprule
& vs & am & carb & wt & drat & gear & cyl & qsec & mpg & hp &
disp\tabularnewline
\midrule
\endhead
Hornet Sportabout & 0 & 0 & 2 & 3 & 3 & 3 & 8 & 17 & 19 & 175 &
360\tabularnewline
Hornet 4 Drive & 1 & 0 & 1 & 3 & 3 & 3 & 6 & 19 & 21 & 110 &
258\tabularnewline
Valiant & 1 & 0 & 1 & 3 & 3 & 3 & 6 & 20 & 18 & 105 & 225\tabularnewline
Mazda RX4 Wag & 0 & 1 & 4 & 3 & 4 & 4 & 6 & 17 & 21 & 110 &
160\tabularnewline
Mazda RX4 & 0 & 1 & 4 & 3 & 4 & 4 & 6 & 16 & 21 & 110 &
160\tabularnewline
Datsun 710 & 1 & 1 & 1 & 2 & 4 & 4 & 4 & 19 & 23 & 93 &
108\tabularnewline
\bottomrule
\end{longtable}

We might also then remove the column medians from the last table to see
the deviations from the column summary (our potential model).

\begin{Shaded}
\begin{Highlighting}[]
\NormalTok{colMedians <-}\StringTok{ }\KeywordTok{apply}\NormalTok{(newtable, }\DataTypeTok{MARGIN =} \DecValTok{2}\NormalTok{, }\DataTypeTok{FUN =}\NormalTok{ median)}
\NormalTok{colMedians <-}\StringTok{ }\KeywordTok{t}\NormalTok{(}\KeywordTok{as.matrix}\NormalTok{(colMedians))}
\NormalTok{colMedians}
\end{Highlighting}
\end{Shaded}

\begin{verbatim}
##       vs  am carb    wt drat gear cyl   qsec mpg  hp  disp
## [1,] 0.5 0.5  1.5 3.045  3.5  3.5   6 17.815  21 110 192.5
\end{verbatim}

\begin{Shaded}
\begin{Highlighting}[]
\CommentTok{# Now form the new table of deviations by sweeping away the colMedians}
\NormalTok{deviations <-}\StringTok{ }\KeywordTok{sweep}\NormalTok{(newtable, }\DataTypeTok{MARGIN =} \DecValTok{2}\NormalTok{, }\DataTypeTok{STATS =}\NormalTok{ colMedians)}

\CommentTok{# Note that in this table, the columns are variates that are all on different scales.}
\CommentTok{# In such cases it makes sense to use different digits for each of the columns.}
\KeywordTok{kable}\NormalTok{(deviations, }\DataTypeTok{digits =} \KeywordTok{c}\NormalTok{(}\KeywordTok{rep}\NormalTok{(}\DecValTok{1}\NormalTok{,}\DecValTok{6}\NormalTok{), }\KeywordTok{rep}\NormalTok{(}\DecValTok{0}\NormalTok{,}\DecValTok{5}\NormalTok{)))}
\end{Highlighting}
\end{Shaded}

\begin{longtable}[]{@{}lrrrrrrrrrrr@{}}
\toprule
& vs & am & carb & wt & drat & gear & cyl & qsec & mpg & hp &
disp\tabularnewline
\midrule
\endhead
Hornet Sportabout & -0.5 & -0.5 & 0.5 & 0.4 & -0.4 & -0.5 & 2 & -1 & -2
& 65 & 168\tabularnewline
Hornet 4 Drive & 0.5 & -0.5 & -0.5 & 0.2 & -0.4 & -0.5 & 0 & 2 & 0 & 0 &
66\tabularnewline
Valiant & 0.5 & -0.5 & -0.5 & 0.4 & -0.7 & -0.5 & 0 & 2 & -3 & -5 &
32\tabularnewline
Mazda RX4 Wag & -0.5 & 0.5 & 2.5 & -0.2 & 0.4 & 0.5 & 0 & -1 & 0 & 0 &
-32\tabularnewline
Mazda RX4 & -0.5 & 0.5 & 2.5 & -0.4 & 0.4 & 0.5 & 0 & -1 & 0 & 0 &
-32\tabularnewline
Datsun 710 & 0.5 & 0.5 & -0.5 & -0.7 & 0.4 & 0.5 & -2 & 1 & 2 & -17 &
-84\tabularnewline
\bottomrule
\end{longtable}

The table shows a variety of deviation patterns. The last two variates
(\texttt{hp} and \texttt{disp}) have very different deviation patterns
from the previous 9, with a notable outlier in the first row. The first
six variates have deviations that are all about \(\pm 0.5\) with again
some notable outliers.

Given that the variates were all on different scales, in Statistics we
might try to make them comparable by standardizing them (note that this
removes the means from the columns and divides by the standard
deviations)

\begin{Shaded}
\begin{Highlighting}[]
\NormalTok{scaledtable <-}\StringTok{ }\KeywordTok{scale}\NormalTok{(newtable)}
\KeywordTok{kable}\NormalTok{(scaledtable, }\DataTypeTok{digits =} \DecValTok{1}\NormalTok{)}
\end{Highlighting}
\end{Shaded}

\begin{longtable}[]{@{}lrrrrrrrrrrr@{}}
\toprule
& vs & am & carb & wt & drat & gear & cyl & qsec & mpg & hp &
disp\tabularnewline
\midrule
\endhead
Hornet Sportabout & -0.9 & -0.9 & -0.1 & 1.0 & -0.6 & -0.9 & 1.6 & -0.7
& -1.0 & 2.0 & 1.6\tabularnewline
Hornet 4 Drive & 0.9 & -0.9 & -0.8 & 0.5 & -0.7 & -0.9 & 0.0 & 0.9 & 0.5
& -0.2 & 0.5\tabularnewline
Valiant & 0.9 & -0.9 & -0.8 & 1.0 & -1.4 & -0.9 & 0.0 & 1.4 & -1.4 &
-0.4 & 0.1\tabularnewline
Mazda RX4 Wag & -0.9 & 0.9 & 1.2 & -0.2 & 0.9 & 0.9 & 0.0 & -0.7 & 0.3 &
-0.2 & -0.6\tabularnewline
Mazda RX4 & -0.9 & 0.9 & 1.2 & -0.8 & 0.9 & 0.9 & 0.0 & -1.1 & 0.3 &
-0.2 & -0.6\tabularnewline
Datsun 710 & 0.9 & 0.9 & -0.8 & -1.4 & 0.8 & 0.9 & -1.6 & 0.3 & 1.3 &
-0.8 & -1.2\tabularnewline
\bottomrule
\end{longtable}


\end{document}
